\documentclass[uplatex,dvipdfmx]{jsarticle}

\input{"preamble.tex"}

\title{ランダウ、リフシッツ「統計物理学」 ノート}
\author{ガオゾウ}

\begin{document}
\maketitle
\section*{概要}
    時間のある時にランダウ、リフシッツ 理論物理学教程の統計物理学を読んでいるが、そこに書いてあることを自分なりにまとめたりまとめなかったりする。

\section{第一章 統計の基礎原理}
    熱力学的な(十分に自由度の大きい)系が(緩和時間に比べて)十分に長い時間放置されると、系は熱力学的平衡に達する。平衡状態における系の振る舞いを予言するにあたり、力学のようなすべての自由度の時間発展を追いかけるのは事実上不可能なので、統計的な扱いを考える必要がある。すなわち、$2s$次元の相空間\footnote{教科書では位相空間となっているがここでは相空間と書く。}$(sは自由度の数)$上の分布関数$\rho(q,p)$を考える必要がある\footnote{$(q,p)$は正確には$(q_1,q_2, \dots ,q_s, p_1, p_2, \dots, p_s)$を表す。以下同様の記法を用いる。}。熱平衡状態における系の諸物理量は、この分布関数をもとにその期待値でもって計算される。

    \subsection{リウヴィユの定理}  
        孤立した系に関して、その分布関数$\rho(q,p)$の持つべき性質について考えよう。
        $\rho(q,p)$を相空間上の「粒子」の分布のようにとらえれば、その粒子の連続の式のようなものが考えられるだろう。通常の連続の式は、
        \begin{align}
            \pdv{\rho}{t}+\div(\rho \vb*{v}) = 0
        \end{align}
        である。定常状態においては$\pdv{\rho}{t}=0$であることから、
        \begin{align}
            \div(\rho \vb*{v}) = 0            
        \end{align}
        となる。今考えているのは$2s$次元相空間上の「粒子」の連続の式だから、これに対応するのは、
        \begin{align}
            \sum_{i=0}^{2s} \pdv{x_i}(\rho \dot{x_i}) = 0
        \end{align}
        ここで、$x_i(i=1,2,\dots, 2s)$は$q_1,q_2, \dots ,q_s, p_1, p_2, \dots, p_s$を表す。もう少し書き下せば、
        \begin{align}
            \sum_{i=0}^s \left\{ \pdv{(\rho \dot{q_i})}{q_i} + \pdv{(\rho\dot{p_i})}{p_i} \right\} &= 0 \\
            \sum_{i=0}^s \left\{ \pdv{\rho}{q_i}\dot{q_i} + \pdv{\rho}{p_i} \dot{p_i} \right\} 
            + \sum_{i=0}^s \rho \left\{ \pdv{\dot{q_i}}{q_i} + \pdv{\dot{p_i}}{p_i}\right\} &= 0 \label{eq:continuous_1}
        \end{align}
        このうち、$\dot{q_i}$や$\dot{p_i}$は運動方程式から決まる。系のハミルトニアンを$H$として、ハミルトンの運動方程式は
        \begin{align}
            \dot{q_i} = \pdv{H}{p_i} \qc \dot{p_i} = -\pdv{H}{q_i}            
        \end{align}
        であるから、これを式\eqref{eq:continuous_1}に代入すれば、式の第二項はゼロになる。一方、式の第一項は$\rho(q_i(t), p_i(t))$の時間に関する全微分であることから、結局
        \begin{align}
            \dv{\rho}{t} = \sum_{i=0}^s \left\{ \pdv{\rho}{q_i}\dot{q_i} + \pdv{\rho}{p_i} \dot{p_i} \right\} = 0
        \end{align}
        がわかる。
        すなわち、分布関数$\rho$は、系の運動の軌跡に沿って変化しないことがわかる。
        
        ここでの議論は、孤立系に関してはもちろんのこと、他の系と相互作用する部分系であったとしても、十分に短い時間の範囲内であれば(その相互作用による影響は無視できるので)適用できることに注意する。このような部分系の状況のことを教科書ではほとんど閉じた系と名付けている。

    \subsection{エネルギーの役割}
        リウヴィユの定理から直ちにわかることは、$\rho(q,p)$(これは$q$と$p$の関数である)が系の運動の軌跡に関して変化しないような特殊な$q,p$の組み合わせでなくてはならないことである。すなわち、$\rho(q,p)$は運動の積分(保存量)の関数でなくてはならない。さらに、この積分は相加的な量でなくてはならないことが次のようにしてわかる。
    
        1つの孤立系を考えよう。その孤立系を二つに分割し、それぞれ系1、系2とする。全系が平衡状態にあるとき、当然系1、系2もそれぞれで熱平衡状態にある。このとき、系1と系2の間には、相互作用があったとしても非常に弱いとみなしうる。実際そのようにしても、系の巨視的な物理量は変わらないだろう\footnote{平衡状態にある気体の箱に、新たな壁を挿入して二つに分ける場合などを考えるとそんな気がしてくる。}。

        このとき、系1と系2は統計的に独立であるとみなせる。すなわち、全系、系1、系2について、それぞれの分布関数を$\rho(q,p), \rho_1(q^{(1)},p^{(1)}), \rho_2(q^{(2)},p^{(2)})$とすると、次が成り立つ。
        \begin{align}
            \rho(q,p) &= \rho_1(q^{(1)},p^{(1)})\rho_2(q^{(2)},p^{(2)}) \\
            \log \rho &= \log \rho_1 + \log \rho_2 \label{eq:dist_func_additive}
        \end{align}
        上で述べたように相互作用は十分弱いとみなしてよいので、系1、系2はほとんど閉じた系とみなすことができる。したがって、前節の議論が適用できる。つまり、リウヴィユの定理より、$\rho,\rho_1,\rho_2$は全て運動の積分(の関数)でなくてはならない。このとき、$\log\rho,\log\rho_1,\log\rho_2$も当然運動の積分(の関数)である。

        よって、式\eqref{eq:dist_func_additive}より、$\log\rho$は相加的な運動の積分でなくてはならないことがわかる。このような運動の積分は、主に次の三つである:エネルギー、全運動量ベクトル、全角運動量ベクトル。
        したがって、$\log\rho_a (a=1,2)$は次のように書き表すことができる。
        \begin{align}
            \log\rho_a = \alpha_a + \beta E_a + \vb*{\gamma P_a} + \vb*{\delta M_a} \label{eq:canonical_dist1}
        \end{align}
        ここで、$E_a,\vb*{P_a},\vb*{M_a}$はそれぞれ系aのエネルギー、全運動量、全角運動量である。また、$\alpha,\beta,\vb*{\gamma},\vb*{\delta}$は定数である。
        式\eqref{eq:dist_func_additive}が成り立つためには、$\beta, \vb*{\gamma},\vb*{\delta}$は、系1,2の両方について同じ値でならなくてはならないことに注意せよ。また、定数$\alpha_a$は$\rho_a$の規格化条件から定まる。

        さて、表式\eqref{eq:canonical_dist1}はほとんど閉じた部分系1,2に関する表式であった。「ほとんど」閉じた系であるがゆえに、エネルギーや運動量、角運動量などがそのほかの部分系とやり取りをして変化しうるから、$\rho_a$の表式中にこれらが現れることに意味がある。各部分系aがあるエネルギーを持つ確率がどの程度であるか、という問題を考えることはきちんとできる。

        しかし、$\rho_a$をかけて得られる$\rho$についてはどうだろうか。全系は「完全に」閉じた系であるから、全系のエネルギー・全運動量、全角運動量は完全に確定してしまっている。したがって、\eqref{eq:canonical_dist1}のように分布関数の運動の積分への依存性を表す式はほとんど役に立たない。唯一式\eqref{eq:canonical_dist1}(の足し算)からわかることは、運動の積分の値が同じであるような$(q,p)$に対しては、$\rho(q,p)$は全て同じ値をとるということである。

        以上の事情\footnote{ここで述べたことは私個人が自分で考えたことで、教科書にはっきり書いてあるわけではないので、読む人は注意。}を踏まえると、$\rho(q,p)$は、与えられた系のエネルギー$E_0$、全運動量$\vb*{P}_0$、全角運動量$\vb*{M}_0$を用いて次のように表してよいことがわかる。
        \begin{align}
            \rho(q,p) = \text{const.}\delta(E-E_0)\delta(\vb*{P}-\vb*{P}_0)\delta(\vb*{M}-\vb*{M}_0) \label{eq:mcdist1}
        \end{align}
        ここでデルタ関数が現れているのは、$\rho$が規格化条件$\int \dd{q}\dd{p} \rho(q,p) = 1$を満たせるようにするためである。

        式\eqref{eq:mcdist1}はミクロカノニカル分布と呼ばれる。

        閉じた系の運動量と角運動量は、系全体としての運動(並進運動、回転運動)に関係しているから、系全体の運動が与えられていれば、系の統計的状態はエネルギーだけに依存する。したがって、統計物理学においてエネルギーは格段に重要な役割を果たすことになる。

        しばしば、運動量と角運動量に関する考察を省くために次のような方法がとられる。系を箱の中に閉じ込め、箱が静止している座標系をとる。すると、もはや運動量や角運動量は運動の積分ではなくなる。したがって、式\eqref{eq:canonical_dist1},\eqref{eq:mcdist1}は次のようになる。

        \begin{align}
            \rho &= \text{const.}\delta(E-E_0) \\
            \log \rho_a &= \alpha_a + \beta E_a            
        \end{align}
        これらが、一般的にミクロカノニカル分布、カノニカル分布と呼ばれているものである。
\end{document}