%パッケージ導入

%\usepackage[dvipdfm,
%  colorlinks=false,
%  bookmarks=true,
%  bookmarksnumbered=false,
%  pdfborder={0 0 0},
%  bookmarkstype=toc]{hyperref}	%目次リンク化
%\AtBeginDvi{\special{pdf:tounicode EUC-UCS2}}	%目次リンク化
\usepackage{amsmath,amssymb,ascmac,fancybox}	%基本パッケージ
\usepackage[dvipdfmx]{color}
\usepackage[dvipdfmx]{graphicx}
\usepackage{url}
\usepackage{siunitx}
\usepackage{layout}		%layout確認用
\usepackage{tcolorbox}		%tcolorboxとその拡張用パッケージ
	\tcbuselibrary{raster,skins}	
	\tcbuselibrary{xparse}
	\tcbuselibrary{breakable}
\usepackage{fancyhdr}		%ページスタイル用パッケージ
\usepackage{subcaption}		%並べた図に異なる見出しを付ける為のパッケージ
%\usepackage[deluxe]{otf}	%超太字などを使うためのパッケージ(失敗した)
\usepackage{mhchem}		%化学式など用パッケージ
%\usepackage{theorem}		%定義、定理、補題環境
\usepackage{bm}
\usepackage{physics}
%パッケージ導入終わり

%ページスタイル宣言
%\pagestyle{fancy}

%ページレイアウト
%\setlength{\textwidth}{\fullwidth}
%\setlength{\evensidemargin}{\oddsidemargin}

%graphicフォルダ指定
\graphicspath{{graphics/}}

%各種マクロ設定
%各種コマンド
\newtcolorbox{nonframe}[1]{breakable,top = 0.5mm, colframe = white, colback = white, colbacktitle = white,coltitle = black,fonttitle = \bfseries\sffamily,title = #1}	%枠なし、タイトル太字ゴシック
%\newcommand{\Vect}[1]{\mbox{\boldmath $#1$}}		%太字斜体ベクトル
\newcommand{\emp}[1]{\textcolor{red}{\textgt{\bf #1}}}	%太字、赤色
\newcommand{\ham}{\hat{H}}	%ハミルトニアン演算子
\newcommand{\gtb}[1]{\textgt{\textbf{#1}}}	%太字ゴシック
\newcommand{\ex}{\underline{\gtb{例}}}		%「例」の字 太字と下線付き
\newcommand{\refer}[1]{(\ref{#1})}		%括弧つき式番号等参照
\newcommand{\partdif}[2]{\frac{\partial #1}{\partial #2}}	%偏微分 \pardif{a}{b}でaをbで偏微分した表記になる
\newcommand{\substi}[2]{\left.#1\right|_{#2}}		%縦線代入
\newcommand{\Vector}[3]{\left( \begin{array}{c} #1 \\ #2 \\ #3 \end{array} \right)}	%ベクトル成分表示


%各種環境
\newenvironment{column}[1]{\begin{boxnote}{\large \textbf{\textgt{コラム #1}}} \vspace{3pt} \\}{\end{boxnote}}	%コラムボックス環境
\newenvironment{matome}{\begin{tcolorbox}[enhanced,title= \gtb{まとめ},		
  	attach boxed title to top left={xshift=3mm, yshift*=-\tcboxedtitleheight/2}]}
{\end{tcolorbox}}												%まとめボックス環境

%定理環境設定
%\theoremstyle{break}			%定理スタイル指定
%\newtheorem{theo}{定理}[chapter]	%定理環境
%\newtheorem{defi}{定義}[chapter]	%定義環境
%\newtheorem{lemm}{補題}[chapter]	%補題環境
%\newenvironment{defbox}[1]{\begin{tcolorbox}[enhanced,breakable] \begin{defi}[#1]}{\end{defi} \end{tcolorbox}}		%定義ボックス
%\newenvironment{thmbox}[1]{\begin{tcolorbox}[enhanced,breakable] \begin{theo}[#1]}{\end{theo} \end{tcolorbox}}		%定理ボックス
%\newenvironment{lembox}[1]{\begin{tcolorbox}[enhanced,breakable] \begin{lemm}[#1]}{\end{lemm} \end{tcolorbox}}	%補題ボックス

