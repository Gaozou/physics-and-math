\documentclass[uplatex,dvipdfmx]{jsarticle}

\input{"preamble.tex"}

\newcommand{\ave}[1]{\langle #1 \rangle}

\title{黒体輻射の計算}
\author{ガオゾウ}

\begin{document}
\maketitle
\section{概要}
ふと思い立って黒体輻射の計算をすることにしました。B2とかで習ったときはいまいち何をやってるのか頭の中で整理しきれなかったので、その整理もかねて。

\section{問題設定}
温度Tの黒体による輻射の周波数スペクトルを求める。

\section{解答に必要な知識の整理}\label{sec:knowledge} 
黒体輻射の計算では、必然的に、光子数の熱力学的な期待値を計算する必要が出てくる。光子はボソンの一つであるので、ボーズ・アインシュタイン分布に従うが、ここではカノニカル分布を用いて光子数の熱力学的な期待値を計算し、これを導出しておくことにする。

\vspace{0.5cm}

真空中の電磁波の一つのモードを考えよう。ここでモードと呼んでいるのは、例えば「波数ベクトル$\vb*{k}$であり$y$方向の偏光の平面波$\vb*{E}e^{i\vb*{kr}}$」など、電磁波の波動方程式を満たす一つの独立な解のことを指す。基本的にはある波数ベクトル$\vb*{k}$と偏光の向きを指定して決まる平面波と思えばよい。

量子論的には、電磁波は量子的である。これは、やや雑に言えば、電磁波の各モードの振幅(あるいは各モードの強度、エネルギー)が、連続的な値ではなく離散的な値しか取れないことを意味する。どのように離散化されているかというと、角周波数$\omega$の電磁波の一つのモードが持つエネルギーは、$\hbar \omega$の整数倍しか許されないように離散化されている\footnote{より正確には$E=(n+1/2)\hbar\omega(nは整数)$であるようなエネルギー$E$しか許されない。今は$n=0$のときのエネルギー(零点エネルギー)をエネルギーの原点を取り換えることで無視している。なお、電磁波のある角周波数$\omega$のモードが、エネルギー$E=(n+1/2)\hbar\omega$を持つとき、「そのモードの光子がn個ある」などと表現される。}。

以上を踏まえて、電磁波の角周波数が$\omega$であるようなあるモードについて、そのモードの持つエネルギーの期待値を計算してみよう。すでに述べたように、計算にはカノニカル分布を用いることにする。

カノニカル分布では、エネルギー$E$を持つ状態は、ボルツマン因子$e^{-\beta E}$に比例する確率で出現する。この確率の規格化定数
\begin{align}
    Z(\beta) = \sum_{状態} e^{-\beta E}    
\end{align}
は分配関数と呼ばれるのだった。今回の電磁波の一つのモードについて、この分配関数をまず求めてみよう。

今回系がとりうる状態は、エネルギー$E=n\hbar\omega(n=0,1,2,\dots)$を持つ状態一つ一つである\footnote{ここではエネルギーの原点をずらし、零点エネルギーは消去した。}から、分配関数は、

\begin{align}
    Z(\beta) &= \sum_{n=0}^{\infty} e^{-\beta n\hbar\omega} \\
    &= \frac{1}{1-e^{-\beta\hbar\omega}}
\end{align}


一つのモードが持つ平均の光子数は次のように計算できる。

\begin{align}
    \ave{n} &= \frac{\sum_{n=0}^{\infty} n e^{-\beta n\hbar\omega}}{Z(\beta)} \\
     &= -\frac{1}{\hbar\omega} \dv{\log{Z}}{\beta} \\
     &= \frac{1}{e^{\beta\hbar\omega}-1} \label{eq:BEdist}
\end{align}

最後の表式\ref{eq:BEdist}は、ボース・アインシュタイン分布と呼ばれる分布において、化学ポテンシャル$\mu=0$とした特別な場合である。化学ポテンシャルが0となっているのは、光子の粒子数は保存せず、いくつでもよいことに対応している。

なお、今注目しているモードの持つエネルギーの期待値は、

\begin{align}
    \ave{E} &= \ave{n\hbar\omega} = \frac{\hbar\omega}{e^{\beta\hbar\omega}-1}  \label{eq:energy}
\end{align}

となる。

\section{解答}
本題に入る前に、もう少し詳細に系の状況設定を詰めておこう。考える状況としておそらく最もイメージしやすいのは、黒体で作られた中空の箱があり、箱の中の壁から電磁波が放出(黒体輻射)されている状況である。

箱は一辺の長さが$L$の立方体で、体積$V=L^3$であるとし、系は温度$T$の熱平衡状態にあるものとする。

このとき考えたいのは、箱の中の電磁波の熱平衡状態である。すなわち、熱平衡状態において、周波数$\nu$(角振動数$\omega=2\pi\nu$)の電磁波がどのくらいの強度で存在しているのかを求めたい。

これを考えるには、考えうる電磁波のすべてのモードに対して、前節で求めたエネルギー$E$の表式\ref{eq:BEdist}, \ref{eq:energy}を適用し、それらの総和を求めればよい。

今、1つのモードに関するエネルギーの表式\ref{eq:energy}は、波数ベクトルの詳細によらず角振動数$\omega$のみに依っているから、同じ角振動数$\omega$を持つモードがいくつあるのかを考えてみよう。

一般に光速$c$、波数ベクトル$\vb*{k}$、角振動数$\omega$の間には、次のような関係がある。
\begin{align}
    |\vb*{k}|=\frac{\omega}{c}    
\end{align}
1つのモードが波数ベクトルと偏光の向きで決まることも考えると、角振動数が$\omega$以下であるようなモードの数$W(\omega)$は、

\begin{align}
    W(\omega) &= \frac{2\times \frac{4\pi}{3} \left(\frac{\omega}{c} \right)^3}{\left( \frac{2\pi}{L} \right)^3} \\
        &= \frac{V\omega^3}{3\pi^2c^3}
\end{align}
ここで、2倍されているのは偏光の向きの自由度を表す。また、$\left( \frac{2\pi}{L} \right)^3$は、波数空間上でモード一つが占めている体積を表す\footnote{これは考えている系が周期的境界条件を課されている場合に厳密に正しい。実際問題として、系の体積が十分に大きければ、境界条件は系の物理に影響しないと考えられるので、物理的にはやや不自然であるが計算の都合上、周期的境界条件を課していると考えればよいと思う。}。

したがって、角振動数が$\omega$から$\omega+\dd{\omega}$の間にあるようなモードの数$D(\omega)\dd{\omega}$は、

\begin{align}
    D(\omega)\dd{\omega} &= \dv{W}{\omega} \dd{\omega} \\    
        &= \frac{V\omega^2}{\pi^2c^3} \dd{\omega}
\end{align}
したがって、角振動数$\omega$~$\omega+\dd{\omega}$を持つモードのエネルギーの総和$U(\omega)\dd{\omega}$は、式\ref{eq:energy}より
\begin{align}
    U(\omega)\dd{\omega} &=  \ave{E}D(\omega)\dd{\omega} \\
        &= \frac{\hbar\omega}{e^{\beta\hbar\omega}-1} \frac{V\omega^2}{\pi^2c^3} \dd{\omega} \\
        &= V\frac{\hbar \omega^3}{\pi^2c^3(e^{\beta\hbar\omega} -1)}
\end{align}
エネルギー密度$u(\omega) = U(\omega)/V$は、
\begin{align}
    u(\omega) &= \frac{\hbar \omega^3}{\pi^2c^3(e^{\beta\hbar\omega} -1)} \\
    \qq{あるいは} u(\nu) &= u(\omega) \dv{\omega}{\nu} \\
        &= \frac{8\pi h \nu^3}{c^3}\frac{1}{e^{h\nu/k_BT} -1} 
\end{align}
これがプランクの輻射公式である。

\section{補足:電磁波の(大雑把な)量子化}
\ref{sec:knowledge}節では「電磁波が量子化されると$\hbar \omega$の整数倍のエネルギーのみをとる」ことを天下り的に認めた。しかし、電磁波が量子化されると$\hbar \omega$の整数倍をとることは、次のような方法で一応納得することもできる。

真空中の電磁波の挙動を決定する方程式は、次のような波動方程式である。

\begin{align}
    \left(\nabla^2 - \frac{1}{c^2}\pdv[2]{}{t}\right) \vb*{E}(\vb*{r},t) &= 0 \\
    \left(\nabla^2 - \frac{1}{c^2}\pdv[2]{}{t}\right) \vb*{B}(\vb*{r},t) &= 0    
\end{align}

このような波動方程式は、空間成分についてフーリエ変換を行うと次のようになる。

\begin{align}
    \left(k^2 + \frac{1}{c^2}\pdv[2]{}{t}\right) \vb*{E}(\vb*{k},t) &= 0 \\
    \left(k^2 + \frac{1}{c^2}\pdv[2]{}{t}\right) \vb*{B}(\vb*{k},t) &= 0
\end{align}
$ck = \omega$を用いて少し変形すると、
\begin{align}
    \pdv[2]{}{t} \vb*{E}(\vb*{k},t) &= -\omega^2 \vb*{E}(\vb*{k},t) \\
    \pdv[2]{}{t} \vb*{B}(\vb*{k},t) &= -\omega^2 \vb*{B}(\vb*{k},t) \\
\end{align}
これらは固有振動数$\omega$の調和振動子に関する運動方程式とまったく同じ形をしている。
これらの式から、電磁波の電場・磁場のフーリエ変換は、各$\vb*{k}$ごとに独立に、調和振動子と同じ形の方程式に従うことがわかる。したがって、量子論においても、電磁波が調和振動子と同様の振る舞いをすることが期待できる。

量子力学において、固有振動数$\omega_0$の調和振動子のエネルギー固有値は$E_n = (n+1/2)\hbar\omega_0$であることから、角振動数$\omega$の電磁波もまったく同様にエネルギー固有値として$E_n = (n+1/2)\hbar\omega$となることも予想できよう。

\vspace{0.5cm}

実際には、電場と磁場が互いに独立でないことやゲージ変換の自由度があることなどから、特に電磁波の持つ自由度についてもう少し詳細な議論が必要となる\footnote{クーロンゲージに固定したベクトルポテンシャルを用いると簡単に議論できる。}が、角周波数$\omega$を持つモードが調和振動子と同様に振る舞うという結論は変わらない。これが、プランクのエネルギー量子仮説に対応する結果である。
\end{document}