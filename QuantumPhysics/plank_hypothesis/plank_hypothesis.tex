\documentclass[uplatex,dvipdfmx]{jsarticle}

\input{"preamble.tex"}

\title{黒体輻射の計算}
\author{ガオゾウ}

\begin{document}
\maketitle
\section{概要}
ふと思い立って黒体輻射の計算をすることにしました。B2とかで習ったときはいまいち何をやってるのか頭の中で整理しきれなかったので、その整理もかねて。

\section{問題設定}
温度Tの黒体による輻射の周波数スペクトルを求める。
\section{解答}
本題に入る前に、もう少し詳細に系の状況設定を詰めておこう。考える状況として最もイメージしやすいのは、黒体で作られた中空の箱があり、箱の中の壁から電磁波が放出されている状況である。

箱は長さ$L$、体積$V=L^3$であるとし、系は温度$T$の熱平衡状態にあるものとする。

このとき、考えるたいのは、箱の中の電磁波の熱平衡状態である。すなわち、熱平衡状態において、周波数$\nu$の電磁波がどのくらいの強度で存在しているのかを求めたい。




\end{document}