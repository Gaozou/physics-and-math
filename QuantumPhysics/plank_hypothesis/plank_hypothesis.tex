\documentclass[uplatex,dvipdfmx]{jsarticle}

\input{"preamble.tex"}

\newcommand{\ave}[1]{\langle #1 \rangle}

\title{黒体輻射の計算}
\author{ガオゾウ}

\begin{document}
\maketitle
\section{概要}
ふと思い立って黒体輻射の計算をすることにしました。B2とかで習ったときはいまいち何をやってるのか頭の中で整理しきれなかったので、その整理もかねて。

\section{問題設定}
温度Tの黒体による輻射の周波数スペクトルを求める。

\section{解答に必要な知識の整理}
黒体輻射の計算では、必然的に、光子数の熱力学的な期待値を計算する必要が出てくる。光子はボソンの一つであるので、ボーズ・アインシュタイン分布に従うが、ここではカノニカル分布を用いて光子数の熱力学的な期待値を計算し、これを導出しておくことにする。

\vspace{0.5cm}

真空中の電磁波の一つのモードを考えよう。ここでモードと呼んでいるのは、例えば「波数ベクトル$\vb*{k}$であり$y$方向の偏光の平面波$\vb*{E}e^{i\vb*{kr}}$」など、電磁波の波動方程式を満たす一つの独立な解のことを指す。基本的にはある波数ベクトル$\vb*{k}$と偏光の向きを指定して決まる平面波と思えばよい。

量子論的には、電磁波は量子化されている。これは、やや雑に言えば、電磁波の各モードの振幅(あるいは各モードの強度、エネルギー)が、連続的な値ではなく離散的な値しか取れないことを意味する。どのように離散化されているかというと、角周波数$\omega$の電磁波の一つのモードは$\hbar \omega$の整数倍のエネルギーに対応する強度しか許されないように離散化されている\footnote{より正確には$E=(n+1/2)\hbar\omega(nは整数)$であるようなエネルギー$E$しか許されない。今は$n=0$のときのエネルギー(零点エネルギー)を無視している。なお、電磁波のある角周波数$\omega$のモードが、エネルギー$E=(n+1/2)\hbar\omega$を持つとき、「角周波数$\omega$の光子がn個ある」などと表現される。}。

以上を踏まえて、電磁波の角周波数が$\omega$であるようなあるモードについて、そのモードの持つエネルギーの期待値を計算してみよう。すでに述べたように、計算にはカノニカル分布を用いることにする。

カノニカル分布では、エネルギー$E$を持つ状態は、ボルツマン因子$e^{-\beta E}$に比例する確率で出現する。この確率の規格化定数
\begin{align}
    Z(\beta) = \sum_{状態} e^{-\beta E}    
\end{align}
は分配関数と呼ばれるのだった。今回の電磁波の一つのモードについて、この分配関数をまず求めてみよう。

今回系がとりうる状態は、エネルギー$E=n\hbar\omega(n=0,1,2,\dots)$を持つ状態一つ一つである\footnote{ここではエネルギーの原点をずらし、零点エネルギーは消去した。}から、分配関数は、

\begin{align}
    Z(\beta) &= \sum_{n=0}^{\infty} e^{-\beta n\hbar\omega} \\
    &= \frac{1}{1-e^{-\beta\hbar\omega}}
\end{align}


一つのモードが持つ平均の光子数は次のように計算できる。

\begin{align}
    \ave{n} &= \frac{\sum_{n=0}^{\infty} n e^{-\beta n\hbar\omega}}{Z(\beta)} \\
     &= -\frac{1}{\hbar\omega} \dv{\log{Z}}{\beta} \\
     &= \frac{1}{e^{\beta\hbar\omega}-1} \label{eq:BEdist}
\end{align}

最後の表式\ref{eq:BEdist}は、ボース・アインシュタイン分布と呼ばれる分布において、化学ポテンシャル$\mu=0$とした特別な場合である。化学ポテンシャルが0となっているのは、光子の粒子数は保存せず、いくつでもよいことに対応している。

なお、今注目しているモードの持つエネルギーの期待値は、

\begin{align}
    \ave{E} &= \ave{n\hbar\omega} = \frac{\hbar\omega}{e^{\beta\hbar\omega}-1}  
\end{align}

となる。

\section{解答}
本題に入る前に、もう少し詳細に系の状況設定を詰めておこう。考える状況としておそらく最もイメージしやすいのは、黒体で作られた中空の箱があり、箱の中の壁から電磁波が放出されている状況である。

箱は一辺の長さが$L$の立方体で、体積$V=L^3$であるとし、系は温度$T$の熱平衡状態にあるものとする。

このとき、考えたいのは、箱の中の電磁波の熱平衡状態である。すなわち、熱平衡状態において、周波数$\nu$(角振動数$\omega=2\pi\nu$)の電磁波がどのくらいの強度で存在しているのかを求めたい。

これを考えるには電磁波のモード

\end{document}