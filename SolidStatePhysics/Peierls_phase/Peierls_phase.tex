\documentclass[uplatex,dvipdfmx, a4j]{jsarticle}

\input{"preamble.tex"}

\title{Tight-Binding ModelにおけるPeierls位相}
\author{ガオゾウ}

\begin{document}
\maketitle
\section{概要}
磁場下の一電子ハミルトニアンにおいては、ふつう運動量演算子$\vb*{p}$を$\vb*{p}-\frac{e}{c}\vb*{A}$で置き換えることによって磁場の効果を取り入れる。これはまあいいのだけれど、これと同じことをTight-Binding modelにおいては、磁場の効果を電子のホッピングの際の係数に位相因子をつけることで取り入れる。この手続きのことをPeierls substitutionとかこの位相のことをPeierls phaseなどと呼ぶらしい。
しかし、このTight-Binding ModelにおけるPeierls位相の導出が、少なくとも日本語ではあまり出てこなかった。唯一出てきたそれらしいものは\cite{NagaiNote}のみだったし、ちょっとこれだけでは個人的にあまり納得がいかなかったので、色々調べて書いてみた。本当にこれで正しいのかはわかりません。ちなみにPeierlsの原著論文見たらドイツ語で死にました。

\section{Tight-Binding Model}
まず初めにTight-Binding Modelを、孤立原子における波動関数から始めてきちんと定義する。以下では簡単のため、結晶を構成する原子は一種類のみとし、また考える孤立原子の原子軌道は一種類のみとするが、これらの拡張はほとんど自明である(添え字を適当につけなおせばよい)。

考える孤立原子の位置を$\vb*{x}_i$、原子軌道の波動関数を$\phi(\vb*{x}-\vb*{x}_i)$と書き、対応する状態ベクトルを$\ket*{i}$とかく。$i$番目の







\begin{thebibliography}{99}
	\bibitem{NagaiNote} \url{http://park.itc.u-tokyo.ac.jp/kato-yusuke-lab/nagai/note_110711_mag.pdf}
\end{thebibliography}


\end{document}