\documentclass[uplatex,dvipdfmx]{jsarticle}

\input{"preamble.tex"}

\title{物質科学のための群論入門}
\author{ガオゾウ}

\begin{document}
\maketitle
\section{導入}
\subsection{物質の対称性と群論}
現実の物質を対象とする物質科学は、理論的な解析が非常に困難なことが多いです。その理由としては例えば、ほとんどの場合に自由度が非常に多いことや、相互作用を考慮しなくてはならないことが挙げられます。このような系では、理論的に厳密な解析をするのは困難であるため、しばしば大胆な近似を行う必要があります。

そんな物質の理論を構築するにあたり非常に重要となる一つの手がかりが、物質の持つ対称性です。対称性のもっとも簡単な例は「左右対称」でしょう。しばしば物質も「左右対称」を含む様々な対称性を持ちえます。実は、このような対称性を持つ・持たないという情報だけから物質の様々な性質を知ることができるのです。

% 例えば、キラル分子が旋光性を持つ(光学活性である)というのは、まさに分子の持つ対称性から理解することができます。そのほか錯体分子の配位子場分裂や結晶場分裂は、その物質の持つ対称性の変化から生じるものです。
% また、固体物理学で最も基本的な理論であるバンド理論も物質の対称性に基づく理論の一つです。本来バンド理論は結晶中の電子の理論ですが、そこで出てくる「分散関係」という言葉がそのほかの様々な結晶中の粒子(フォノン、マグノン、etc.)にも用いられているのは、これらの理論は全て共通して結晶の持つ対称性を背景に持つからです。

% いろいろと例を挙げましたが、物理・化学と分野を問わず、物質科学において「対称性」が非常に基本的な考え方であることがわかっていただけたでしょうか。

これらのような対称性の議論をするにあたって、非常に強力な道具となるのが本記事のテーマである群論、特に群の表現論です。

%  対称性のもっとも簡単な例の一つは正方形です。正方形は様々な対称性を持っています。例えば、正方形は図1(a)のように、中心線に関して折りたたむときれいに重なります。つまり左右対称です。このことは、

 

% また、図1(b)のように90°(1/4回転)だけ正方形を回転しても、元の形に戻ります。このような対称性を4回対称性と呼びます。正方形は同時に、2回対称性も持ちます。すなわち、180°(1/2回転)だけの回転に対しても、やはり元の形に戻ります。

物質科学における対称性の重要性

対称性を扱う学問としての群論



\end{document}