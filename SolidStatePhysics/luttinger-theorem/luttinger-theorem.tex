\documentclass[uplatex,dvipdfmx]{jsarticle}

\input{"preamble.tex"}

\title{Luttingerの定理}
\author{ガオゾウ}

\begin{document}
\maketitle

\section{はじめに}
    院試勉強をしていたら、逆格子空間上での準位の数の数え方がよくわからなくなったので整理してみる。これを定理の形でまとめているのがいわゆるLuttingerの定理である。

\section{Blochの定理と逆格子空間}
	まずBlochの定理と逆格子空間について整理する。
	
	結晶のような(離散的な)並進対称性のある系においては、ハミルトニアンが$\vb{R}$だけの並進操作を表すユニタリ変換$\hat{T}_{\vb{R}}$と可換である。よってエネルギー固有状態として$\hat{T}_{\vb{R}}$との同時固有状態をとることが出来る。

	したがって、エネルギー固有状態を区別するラベル(量子数)の一つとして、演算子$\hat{T}_{\vb{R}}$に対してエネルギー固有状態が持つ固有値を用いることができる。より正確には、並進ベクトル$\vb{R}$に対して、エネルギー固有状態が持つ固有値は$e^{i\vb{k}\vdot\vb{R}}$となるので、この$\vb{k}$を用いて固有状態(の集合)を一つ指定することができる。

	このベクトル$\vb{k}$を波数ベクトルといい、このベクトルが張る空間を逆格子空間や波数空間などと呼ぶ。

	以上をもう少しきちんとまとめたものが以下に述べるBlochの定理である。Blochの定理を述べる前に、結晶の周期性などをきちんと扱うための数学的な準備をする。$d$次元結晶の周期性は、d本の基本並進ベクトル$\vb{a}_1, \vb{a}_2, \dots , \vb{a}_d$で表される。結晶は、基本並進ベクトルやそれらの整数倍の足し合わせ$\vb{R} = n_1\vb{a}_1 + n_2\vb{a}_2 + \dots + n_d\vb{a}_d$だけの並進に対して不変である。
	
	また、波数ベクトルを扱う際に便利な逆格子ベクトルを導入しておく。基本逆格子ベクトルは、$d$本の基本並進ベクトルの組$\vb{a}_1, \vb{a}_2, \dots , \vb{a}_d$に対して次のように定義される。基本逆格子ベクトル$\vb{b}_1, \dots, \vb{b}_d$とは、$\vb{a_i}\vdot\vb{b_j} = 2\pi\delta_{ij}$となるようなベクトルのことである\footnote{$\vb{a_i}\vdot\vb{b_j} = \delta_ij$と定義する流儀もある。}。また、逆格子ベクトルとは、基本逆格子ベクトルの整数倍の和で表されるベクトルのことである。すなわち、$\vb{G} = m_1\vb{b}_1 + \dots + m_d\vb{b}_d$と書かれるようなベクトル$\vb{G}$のことである。

	以上を用いてBlochの定理を以下に述べる。ただし、以下では簡単のため$d=3$の場合について述べる。

	\begin{itembox}[l]{Blochの定理}
		基本並進ベクトルが$\vb{a}_1, \vb{a}_2, \vb{a}_3$であるような並進対称性を持つ系を考える。対応する基本逆格子ベクトルを$\vb{b}_1,\vb{b}_2, \vb{b}_3$とする。

		このとき、系のエネルギー固有状態はある波数ベクトル$\vb{k}$を用いて、次のような性質を持つように選べる。
		\begin{align}
			\psi(\vb{r}+\vb{R}) = e^{i\vb{k}\vdot\vb{R}} \psi(\vb{r}) 
		\end{align}

		特に、周期境界条件
		\begin{align}
			\psi(\vb{r}) = \psi(\vb{r}+\vb{N_i\vb{a}_i}) \qq{($i=1,2,3, N_iは整数$)}
		\end{align}
		の下では、
		\begin{align}
			\vb{k} = \frac{m_1}{N_1}\vb{b}_1+\frac{m_2}{N_2}\vb{b}_2 + \frac{m_3}{N_3}\vb{b}_3 \qq{$m_i$は整数}	
		\end{align}
		と書ける。
	\end{itembox}

	\section{準位の数と電子数の関係}
	Blochの定理を踏まえて、準位の数と電子数の関係について議論していこう。簡単のため、以下ではバンドが一つである場合を考える。また、系には周期境界条件が課せられているものとする。

	このとき、第一ブリルアンゾーン(以下BZ)に含まれる、エネルギー固有状態に対応する波数ベクトルの数は$N_1N_2N_3$個である。したがって、$N_i$が十分大きい時には、逆格子空間上の単位体積当たりの準位数$\tilde{\rho}_s$は次のように見積もることができる。
	\begin{align}
		\tilde{\rho}_s = 2\frac{N_1N_2N_3}{\tilde{V}_{BZ}} = 2\frac{N_1N_2N_3}{|\vb{b}_1\vdot(\vb{b}_2\cp\vb{b}_3|}	\label{eq:dos_k}
	\end{align}
	ただしここで$\tilde{V}_{BZ}$はBZの占める体積\footnote{このpdfでは逆格子空間上での量にはチルダをつけて表すことにする。}である。スピン自由度を考慮して因子2がついている。

	なお、逆格子空間におけるBZの体積$\tilde{V}_{BZ}$は、実空間上の単位胞の体積$V_{UC}$と次のように対応している\footnote{証明は、基本並進ベクトルを縦に並べた行列$A$と基本逆格子ベクトルを横に並べた行列$B$が$B=2\pi A^{-1}$の関係にあることと、$V_{UC}=|\det(A)|, \tilde{V}_{BZ} = |\det(B)|$であることから直ちに従う。}:
	\begin{align}
		\tilde{V}_{BZ} = \frac{(2\pi)^3}{V_{UC}}		
	\end{align}
	したがって、式\eqref{eq:dos_k}は次のようにも書ける。
	\begin{align}
		\tilde{\rho}_s = 2\frac{N_1N_2N_3V_{UC}}{(2\pi)^3} = 2\frac{V}{(2\pi)^3} \label{eq:dos_k_2}
	\end{align}
	ここで$V$は実空間における結晶の体積である。
	
	したがって、単位胞当たりの電子の数を$n$、電子の数密度を$\rho = n/V_{UC}$とすると、フェルミ面で囲まれている逆格子空間上の体積(すなわち電子に占有されている部分の体積)を$\tilde{V}_F$とすると、次が成り立つ。
	\begin{align}
		\tilde{\rho}_s\tilde{V}_F = nN_1N_2N_3 = \rho V 
	\end{align}
	これを式\eqref{eq:dos_k}や\eqref{eq:dos_k_2}を用いて整理すると次のようになる。
	\begin{align}
		\tilde{V}_F &= (2\pi)^3 \frac{\rho}{2}
	\end{align}
	あるいは
	\begin{align}
		\frac{\tilde{V}_F}{\tilde{V}_{BZ}} &= \frac{n}{2} \label{eq:luttinger}
	\end{align}
	この式\eqref{eq:luttinger}がいわゆるLuttingerの定理と呼ばれている表式である。
	すなわち、逆格子空間上では、フェルミ面の囲む体積がBZに占める割合は、単位胞あたりの電子数nに比例することがわかる。式\eqref{eq:luttinger}で2がついているのはスピン自由度が原因である。
\end{document}