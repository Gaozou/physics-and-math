\documentclass[uplatex,dvipdfmx]{jsarticle}

\input{"preamble.tex"}

\title{統計力学第二 まとめ}
\author{ガオゾウ}

\begin{document}
\maketitle
\section{BoltzmannのH定理}
エントロピーの最大の特徴は不可逆な時間変化をするという点である。エントロピーが増加することを、ミクロな視点から考察したものとして、以下に述べるBoltzmannのH定理と呼ばれるものがある。

以下では古典系で考える。古典粒子の体積当たりの分布関数$f(r,p,t)$\footnote{この節ではたびたびベクトルrやpを太字などにせずに表記する。}について、次のような量 H関数を考える。

\begin{align}
    H &= \int drdp f(r,p,t) \log f(r,p,t)
\end{align}

この関数が時間に伴いどのように変化するかを考えよう。
Hを時間微分すると、
\begin{align}
    \frac{dH}{dt} &= \int drdp \frac{df}{dt}(1+\log f(r,p,t)) \label{eq:dHdt}
\end{align}
と書ける。そこで、$df/dt$について考えよう。

まず、粒子間に相互作用がなく、各粒子は互いに独立に運動している場合を考える。このとき、各粒子の運動量は保存する。今考えたいのは$f(r,p,t)$の時間発展、すなわち状態が$(r,p)$であるような粒子の数がどのように変化していくかであるが、運動量保存であることから、$f(r,p,t)$の変化を追うには、同じ運動量$p$を持つ粒子のみを考えればよい。

そこで、運動量$p$を一つ固定する。このとき、$f(r,p,t)$の微小時間dtの間の時間変化は、連続の式(粒子数の保存)より次のように書くことができる。
\begin{align}
    \frac{df}{dt} = - \div \vb{j}(r,p) 
\end{align}
ここで、$\vb{j}(r,p)$は位置$r$、運動量$p$である粒子の流れを表す。
さらに、$\vb{j}(r,p)$は次のように書くことができる。
\begin{align}
    \vb{j}(r,p) &= \vb{v} f(r,p,t)
\end{align}
ここで、$\vb{v}$は運動量$p$の粒子の速度である。結局、
\begin{align}
    \frac{df}{dt} = - \div [\vb{v}f(r,p,t)] \label{eq:dfdt_no_int}
\end{align}
であることがわかる。以上の議論は、各粒子の運動量が保存しているために、$f(r,p,t)$の変化を考えるうえでは$f(r',p,t)$のような同じ$p$についてのみ考えればよいことから成り立つことに注意せよ。

さて、求まった$df/dt$をH関数の微分の式に代入すると、

\begin{align*}
    \frac{dH}{dt} &= - \int drdp \div[\vb{v}f(r,p,t)](1+\log f(r,p,t)) \\
        &= - \int drdp \div[\vb{v}f(r,p,t)](1+\log f(r,p,t)) \\
        &= - \int drdp \frac{1}{m} \div[\vb{p}f(r,p,t) \log f(r,p,t)] \\
        &= - \int dSdp \frac{1}{m} \vb{p}f(r,p,t) \log f(r,p,t)
\end{align*}
最後の行では、ガウスの発散定理を用いた。

さて、最後の表面積分で、積分範囲を十分大きく取り、表面S付近には粒子がいないような状況を考えれば、これはゼロになる。したがって、相互作用がまったくない場合には、$\frac{dH}{dt}=0$であることがわかる。

\vspace{1cm}

次に、粒子間の相互作用がある場合を考えよう。この場合、以上までの議論とはことなり、各粒子の運動量は一般に保存せず、粒子間で運動量のやり取りが生じる。この場合には、$df/dt$は上記式\ref{eq:dfdt_no_int}から次のような変更を受ける。

\begin{align}
    \frac{df(r,p,t)}{dt} &= - \div [\vb{v}f(r,p,t)] + \int dp_1dp_2dp_3 \sigma(p_1,p_2,p_3,p) \{f(r,p_1,t)f(r,p_2,t)-f(r, p_3, t)f(r, p,t)\}
\end{align}
ここで、$\sigma(p_1,p_2,p_3,p)$は運動量$p_1,p_2$を持つ二つの粒子が相互作用(衝突)し、運動量がそれぞれ$p_3,p$となる散乱の衝突断面積である。散乱は可逆だから、この逆過程(すなわち運動量$p_3,p$を持つ二つの粒子が相互作用(衝突)し、運動量がそれぞれ$p_1,p_2$となる散乱)も同じ散乱断面積を持つことを用いている。
さらに、$\sigma(p_1,p_2,p_3,p) \propto \delta((p_1+p_2)-(p_3+p))$(運動量保存)であることや、同様にしてエネルギー保存則を用いるともう一つ変数が減らせることを用い、式\ref{eq:dHdt}に代入すると、

\begin{align}
    \frac{dH}{dt} &= \int drdp \frac{df}{dt}(1+\log f(r,p,t))\\
        &= \int drdp \int dp_3 \{\sigma(p_1,p_2,p_3,p) \{f(r,p_1,t)f(r,p_2,t)-f(r, p_3, t)f(r, p,t)\}(1+\log f(r,p,t))\} \label{eq:dHdt_2}
\end{align}

が得られる。ただし、前述のようにガウスの発散定理を用いて1つ項を落としてある。

ここで系の対称性より、式\ref{eq:dHdt_2}において積分変数$p,p_3$の入れ替えを行っても値は変わらないはずである。また、$p_1,p_2$の組と$p,p_3$の組を入れ替えても、やはり式\ref{eq:dHdt_2}の値は変わらないはずである。さらに、$p_1,p_2$の組と$p,p_3$の組を入れ替えたのち、$p, p_3$を入れ替えたものも同じ値である。よって、

\begin{align}
    \frac{dH}{dt} &= \int drdp \int dp_3 \sigma(p_1,p_2,p_3,p) \{f(r,p_1,t)f(r,p_2,t)-f(r, p_3, t)f(r, p,t)\}(1+\log f(r,p,t)) \\ 
        &= \int drdp \int dp_3 \sigma(p_1,p_2,p,p_3) \{f(r,p_1,t)f(r,p_2,t)-f(r, p, t)f(r, p_3,t)\}(1+\log f(r,p_3,t)) \\
        &= \int drdp \int dp_3 \sigma(p_3,p,p_1,p_2) \{f(r,p_3,t)f(r,p,t)-f(r, p_1, t)f(r, p_2,t)\}(1+\log f(r,p_2,t)) \\
        &= \int drdp \int dp_3 \sigma(p,p_3,p_1,p_2) \{f(r,p_3,t)f(r,p,t)-f(r, p_2, t)f(r, p_1,t)\}(1+\log f(r,p_1,t)) \\
        &= \frac{1}{4} \int drdpdp_3 \sigma(p,p_3,p_1,p_2)
            \{ f(r,p_1,t)f(r,p_2,t) - f(r,p_3,t)f(r,p,t) \} \\
        & \hspace{1cm} \times \{\log f(r,p,t)f(r,p_3,t)- \log f(r,p_1,t)f(r,p_2,t)\}
\end{align}

最後の行は、上の四式を足し合わせて4で割ったものを整理することで得られる。この最後の式の被積分関数は、
\begin{align*}
    (a-b)(\log a-\log b) &\geq 0 
\end{align*}
であることと$\sigma$が正であることから、結局
\begin{align}
    \frac{dH}{dt} \leq 0    
\end{align}
がわかる。すなわち、$H(t)$は単調減少である。これがBoltzmannのH定理と呼ばれるものである。

エントロピーは$S=-H$で表されるから、エントロピーが単調増加であることもわかる。

また、定常状態に落ち着く($dH/dt = 0$)となるための十分条件は、$f(r,p_1,t)f(r,p_2,t) - f(r,p_3,t)f(r,p,t)=0 $となることであるが、運動量保存$p_1+p_2 = p_3 + p$と同種粒子の場合のエネルギー保存則$p_1^2+p_2^2 = p_3^2+p^2$を用いると、$f(r,p,t)\propto \exp(Ap+Bp^2)$の形の分布関数は$dH/dt=0$を満たしていることがわかる。さらに空間反転対称性も課せば、$A=0$もわかる。これはMaxwell-Boltzmann分布の形と一致している。
\end{document}